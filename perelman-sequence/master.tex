\documentclass[uplatex, 11pt, a4j, dvipdfmx]{jsarticle}

%------------------------------------------------------------------------
% プリアンブル読み込み
%------------------------------------------------------------------------
%------------------------------------------------------------------------
% usepackages
%------------------------------------------------------------------------
% 使用可能な色を増やす
\usepackage[dvipsnames,svgnames,x11names,dvipdfmx,table,xcdraw]{xcolor}
% 画像読込みを可能に
\usepackage[dvipdfmx]{graphicx}
% レイアウト変更用
\usepackage{geometry}
% 出力用紙サイズのみの変更
\usepackage{bxpapersize}
% セクションのサイズ等を変更
\usepackage{titlesec}
% テーブルのセル内で改行を可能に
\usepackage{tabularx}
% 画像の位置を書いた位置に矯正
\usepackage{here}
% url の表示を可能に
\usepackage{url}
% ソースコード表示用
\usepackage{listings, jlisting}
% otf フォントを使用可能に
\usepackage[uplatex]{otf}
% フォントエンコーディングをT1に変更し、欧文フォントのデフォルトを Latin Modern に変える
\usepackage[T1]{fontenc}
\usepackage{lmodern}

% 図を複数個同一環境にならべたときにキャプションを個別につけられるように
\makeatletter
\let\MYcaption\@makecaption
\makeatother
\usepackage{subcaption}
\captionsetup{compatibility=false}
\makeatletter
\let\@makecaption\MYcaption
\makeatother

% 数式系
\usepackage{amsmath, amssymb}
\usepackage{diffcoeff}
\usepackage{newtxmath, newtxtext}
\usepackage{bm}
\usepackage{ascmac}
% hyperlink
\usepackage[dvipdfmx, bookmarkstype=toc, colorlinks=true, linkcolor=cyan, pdfborder={0 0 0}, bookmarks=true, bookmarksnumbered=true, hypertexnames=false]{hyperref}
\usepackage{pxjahyper}
% 参照を便利に
\usepackage{cleveref}
% 参照した数式のみに数式番号をつける
\usepackage{autonum}
% 数学系いろいろ
\usepackage{mathtools}

% SI単位用
\usepackage{siunitx}                    % SI



%------------------------------------------------------------------------
% setting for listing
%------------------------------------------------------------------------
\lstset{%
    language={C},
    basicstyle={\small},%
    identifierstyle={\small},%
    commentstyle={\small\itshape},%
    keywordstyle={\small\bfseries},%
    ndkeywordstyle={\small},%
    stringstyle={\small\ttfamily},
    frame={tb},
    breaklines=true,
    columns=[l]{fullflexible},%
    numbers=left,%
    xrightmargin=0zw,%
    xleftmargin=3zw,%
    numberstyle={\scriptsize},%
    stepnumber=1,
    numbersep=1zw,%
    lineskip=-0.5ex%
}
\lstset{%
    language={tex},
    basicstyle={\small},%
    identifierstyle={\small},%
    commentstyle={\small\itshape},%
    keywordstyle={\small\bfseries},%
    ndkeywordstyle={\small},%
    stringstyle={\small\ttfamily},
    frame={tb},
    breaklines=true,
    columns=[l]{fullflexible},%
    numbers=left,%
    xrightmargin=0zw,%
    xleftmargin=3zw,%
    numberstyle={\scriptsize},%
    stepnumber=1,
    numbersep=1zw,%
    lineskip=-0.5ex%
}


%------------------------------------------------------------------------
% change the numbering \equation with (section.equation)
%------------------------------------------------------------------------
\makeatletter
    \renewcommand{\theequation} {%
        %\arabic{chapter}.\arabic{section}.\arabic{equation}}
        \arabic{section}.\arabic{equation}}
    \@addtoreset{equation}{section}
\makeatother



\crefname{figure}{図}{図}
\crefname{subfigure}{図}{図}
\crefname{table}{表}{表}
\crefname{subtable}{表}{表}
\crefname{equation}{式}{式}

\newtheorem{dfn}{定義}
\crefname{dfn}{定義}{定義}

\newtheorem{thm}{定理}
\crefname{thm}{定理}{定理}

\newtheorem{lem}{補題}
\crefname{lem}{補題}{補題}

\newtheorem{coro}{系}
\crefname{coro}{系}{系}

\newtheorem{prop}{命題}
\crefname{prop}{命題}{命題}

%\newtheorem{proof}{証明}
\crefname{proof}{証明}{証明}

\newtheorem{ex}{例}
\crefname{ex}{例}{例}

\newtheorem{rem}{注意}
\crefname{rem}{注意}{注意}

\crefformat{page}{#2#1#3{ページ}}%
\crefrangeformat{page}{#3#1#4{〜}#5#2#6{ページ}}%
\crefmultiformat{page}{#2#1#3{ページ}}%
        {, #2#1#3{ページ}}{, #2#1#3}{, #2#1#3{ページ}}%
\crefformat{section}{#2#1#3{節}}%
\crefrangeformat{section}{#3#1#4{〜}#5#2#6{節}}%
\crefmultiformat{section}{#2#1#3{節}}%
        {, #2#1#3{節}}{, #2#1#3}{, #2#1#3{節}}%
\crefformat{subsection}{#2#1#3{節}}%
\crefrangeformat{subsection}{#3#1#4{〜}#5#2#6{節}}%
\crefmultiformat{subsection}{#2#1#3{節}}%
        {, #2#1#3{節}}{, #2#1#3}{, #2#1#3{節}}%
\crefformat{subsubsection}{#2#1#3{節}}%
\crefrangeformat{subsubsection}{#3#1#4{〜}#5#2#6{節}}%
\crefmultiformat{subsubsection}{#2#1#3{節}}%
        {, #2#1#3{節}}{, #2#1#3}{, #2#1#3{節}}%

\crefformat{part}{{第}#2#1#3{部}}%
\crefformat{chapter}{{第}#2#1#3{章}}%

% fig
\newcommand{\fig}[5]{\begin{figure}[#4]\begin{center}\includegraphics[#5]{#1}\caption{#2}\label{fig:#3}\end{center}\end{figure}}

% tab
\newenvironment{tab}[4]{\begin{table}[#1]\centering \caption{#2}\label{tab:#4}\begin{tabular}{#3}}{\end{tabular}\end{table}}

% set
\def\Set#1{\Setdef#1\Setdef}
\def\Setdef#1|#2\Setdef{\left\{#1\,\;\mathstrut\vrule\,\;#2\right\}}%

% sets
\newcommand{\N}{\mathbb{N}}
\newcommand{\Z}{\mathbb{Z}}
\newcommand{\R}{\mathbb{R}}
\newcommand{\Q}{\mathbb{Q}}
\newcommand{\C}{\mathbb{C}}



%------------------------------------------------------------------------
% change font size of section and subsection title
%------------------------------------------------------------------------
%\titleformat*{\section}{\normalsize\bfseries}
%\titleformat*{\subsection}{\normalsize\bfseries}

%------------------------------------------------------------------------
% other setting
%------------------------------------------------------------------------
%\setlength{\columnsep}{3zw}
\setcounter{tocdepth}{3}                       % chenge the depth displayed in the table of contents




\title{ペレリマン数列}
\author{}
\date{}

% -----------------------------------------------------------------------
% 本文
% -----------------------------------------------------------------------

\begin{document}

\maketitle

\section{概要}
  ペレリマン数列は加藤文元、仲井保行``著の天に向かって続く数''という本に出てくる数列である。
  この数列$p_n \quad (0 \leq n)$は
  \begin{equation}
    (p_n)^2 \equiv p_n \quad (mod 10^{n+1}) \label{eq:s}
  \end{equation}
  となる整数の列である。

\section{導出}
  $1$以上の$n$に対して$p_n$を導出する手順を考える。

  まず、ある数が$p_n$をある整数$d_1, d_2$(ただし$0 \leq d_1 < 10, 0 \leq d_2 < 10^n)$によって
  \begin{equation}
    p_n = d_1 \times 10^n + d_2
  \end{equation}
  と書くことができるとする。
  すると$(p_n)^2$は
  \begin{equation} \begin{aligned}
    (p_n)^2 &= (d_1 \times 10^n + d_2)^2 \\
            &= (d_1)^2 \times 10^{2n} + 2 d_1 d_2 \times 10^n + (d_2)^2 \label{eq:aster}
  \end{aligned} \end{equation}
  となる。
  ここである$0$以上の整数$q$と$0$以上$10^n$未満の整数$r$によって
  \begin{equation}
    (d_2)^2 = q \times 10^n + r
  \end{equation}
  とすれば\cref{eq:aster}は
  \begin{equation} \begin{aligned}
    (d_1)^2 \times 10^{2n} + 2 d_1 d_2 \times 10^n + (d_2)^2
            &= (d_1)^2 \times 10^{2n} + 2 d_1 d_2 \times 10^n + q \times 10^n + r \\
            &= (d_1)^2 \times 10^{2n} + (2 d_1 d_2 + q) \times 10^n + r \label{eq:aster2}
  \end{aligned} \end{equation}
  と変形することができる。

  さらに$0$以上の整数$p$と$0$以上$10$未満の整数$s$によって
  \begin{equation}
    2 d_1 d_2 + q = p \times 10 + s
  \end{equation}
  と置くと\cref{eq:aster2}は
  \begin{equation} \begin{aligned}
    (d_1)^2 \times 10^{2n} + (2 d_1 d_2 + q) \times 10^n + r
            &= (d_1)^2 \times 10^{2n} + (p \times 10 + s) \times 10^n + r \\
            &= (d_1)^2 \times 10^{2n} + p \times 10^{n+1} + s \times 10^n + r \\
            &= \left[ (d_1)^2 \times 10^{n-1} + p \right] \times 10^{n+1} + s \times 10^n + r \label{eq:aster3}
  \end{aligned} \end{equation}
  と書くことができる。

  よって
  \begin{equation} \begin{aligned}
    (p_n)^2 = \left[ (d_1)^2 \times 10^{n-1} + p \right] \times 10^{n+1} + s \times 10^n + r
                                                                    \equiv s \times 10^n + r \quad (mod 10^{n+1})
  \end{aligned} \end{equation}
  であり、 $s$と$r$の範囲から\cref{eq:aster3}を満たす$s$と$r$は一意に決まる。
  したがって\cref{eq:s}を満たすためには
  \begin{equation}
    s = d_1, \quad r = d_2
  \end{equation}
  であることが必要十分条件となる。
  そして$r = d_2$であるとき
  \begin{equation}
    (d_2)^2 = q \times 10^n + d_2 \equiv d_2 \quad (mod 10^n)
  \end{equation}
  であるため$d_2 = p_{n-1}$となる。

  \subsection{手順}
    $p_n$を求める手順を以下に示す。
    \begin{enumerate}
      \item $p_{n-1}$を求める
      \item $(p_{n-1})^2 = q \times 10^n + p_{n-1} \rightleftharpoons (p_{n-1})^2 - p_{n-1} = q \times 10^n$を満たす$q$を求める。
      \item $q = 10 p + (1 - 2 p_{n-1}) d$を満たす$p, d$の組のうち、$0 \leq d < 10$となる組み合わせを求める。
      \item $p_n = d \times 10^n + p_{n-1}$とすれば$p_n$が得られ、このとき$(p_n)^2 = \left( d^2 \times 10^{n-1} + p \right) \times 10^{n+1} + d \times 10^n + p_{n-1}$となる
    \end{enumerate}

\end{document}

