\documentclass[uplatex, 11pt, a4j, dvipdfmx]{jsarticle}

%------------------------------------------------------------------------
% プリアンブル読み込み
%------------------------------------------------------------------------
\input{./preembles/all.tex}


%------------------------------------------------------------------------
% ちーとしーと
%------------------------------------------------------------------------
% \section{...}, \subsection{...}, \subsubsection{...}

% \eqref{...} : 数式参照
% \begin{equation}
%     ... \label{eq:...}
% \end{equation}

% \begin{equation}
%     \begin{aligned}
%         ... & ... \\
%         ... & ...
%     \end{aligned} \label{eq:...}
% \end{equation}

% \begin{equation}
%     \left\{ \begin{aligned}
%         ... & ... \\
%         ... & ...
%     \end{aligned} \right. \label{eq:...}
% \end{equation}

% \qquad, \quad : 空白

% \hypertarget{name}{text}
% \hyperlink{name}{text}


% \myTerm{a}{b}   ->    a(b)
\newcommand*{\myTerm}[2]{\textsf{#1}{(\emph{#2})}}

% \begin{dfn}\end{dfn}   ->    definition
\newtheorem{dfn}{定義}

% \begin{thm}\end{thm}   ->    theorem
\newtheorem{thm}{定理}

% \begin{prf}\end{prf}   ->    proof
\newtheorem{prf}{証明}

% \fig
\newcommand{\fig}[5]{\begin{figure}[#4]\begin{center}\includegraphics[#5]{#1}\caption{#2}\label{fig:#3}\end{center}\end{figure}}
% 
\newcommand{\figref}[1]{図\ref{fig:#1}}

%
\newenvironment{tab}[4]{\begin{table}[#1]\centering \caption{#2}\label{tab:#4}\begin{tabular}{#3}}{\end{tabular}\end{table}}
%
\newcommand{\tabref}[1]{表\ref{tab:#1}}



% -----------------------------------------------------------------------
% 本文
% -----------------------------------------------------------------------
\begin{document}
  \begin{dfn}[イデアル]
    $R$を可換環とする。以下を見たす$R$の部分集合$I$を$R$上のイデアルという。
    \begin{description}
      \item[(I1)] $\forall x, y\in I, x + y \in I$
      \item[(I2)] $\forall x\in I, a \in R, a x \in I$
    \end{description}
  \end{dfn}

  \begin{dfn}[剰余類]
    可換環$R$上のイデアル$I$と任意の$x \in R$に対して
    \begin{equation}
      x + I = \{ x + a \mid a \in I \}
    \end{equation}
    を$x$を代表元とする剰余類という。
  \end{dfn}

  \begin{thm}
    ある$x, y \in R$が$x - y \in I$であるとき、$x + I = y + I$である。
    \begin{prf}
      $I$の生成元が$m$であるとする。
      まず定義より$x + I$の任意の元$a$はある$q \in R$によって
      \begin{equation}
        a = x + m q
      \end{equation}
      と書くことができる。
      ここで仮定より、ある$p$によって$x - y = m p \Leftrightarrow x = y + m p$となる。
      この$x$を代入すると
      \begin{align}
        a &= y + m p + m q \\
          &= y + m (p + q)
      \end{align}
      となり、$a \in y + I$が得られた。ここで$a$は任意の$x + I$の元であるため$a \in x + I \Rightarrow a \in y + I$が示された。
      同様にして$a \in y + I \Rightarrow a \in x + I$が示される。
      以上より示された。
    \end{prf}
  \end{thm}



  \begin{dfn}[剰余環]
    可換環$R$上のイデアル$I$に対して
    \begin{equation}
      R/I = \{ x + I \mid x \in R \}
    \end{equation}
    として書かれる集合$R/I$を$R$上の剰余環という。
  \end{dfn}

  \begin{thm}
    可換環$R$上の剰余環$R/I$は可換環となる。
    \begin{prf}
    \end{prf}
  \end{thm}


  \begin{thm}
  \end{thm}


  \begin{thm}[中国剰余定理(Chinese Remainder Theorem)]
    ある剰余環$Z/mZ$に
  \end{thm}


\end{document}

