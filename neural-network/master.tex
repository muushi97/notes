\documentclass[uplatex, 11pt, a4j, dvipdfmx]{jsarticle}

%------------------------------------------------------------------------
% プリアンブル読み込み
%------------------------------------------------------------------------
\input{./preambles/all.tex}


\title{ニューラルネットワーク (ISBN 4-254-11612-8) 自習ノート}
\author{}
\date{}

% -----------------------------------------------------------------------
% 本文
% -----------------------------------------------------------------------

\begin{document}

  \maketitle

  \tableofcontents

  \section{ニューラルネットワークとは何か}
    \subsection{生物に学ぶ}
      \subsubsection{蚊と蟻とサッカーロボット}
      \subsubsection{神経細胞の構造と機能}
    \subsection{神経細胞のモデル}
    \subsection{シナプスの可塑性}
    \subsection{ニューラルネットワークの分類}
      \subsubsection{階層型ニューラルネットワーク}
      \subsubsection{相互結合型ニューラルネットワーク}
    \subsection{ニューラルネットワークの特徴}
      \subsubsection{並列分散処理}
      \subsubsection{学習と自己組織化}

  \section{階層型ニューラルネットワークの情報処理}
    \subsection{パーセプトロン} % 担当1
      \subsubsection{単純パーセプトロン}
      \subsubsection{単純パーセプトロンの学習}
    \subsection{バックプロパゲーション}
      \subsubsection{一般化デルタ則}
      \subsubsection{バックプロパゲーション}
      \subsubsection{応用例} % 担当2
      \subsubsection{ニューラルネットワークの構造とパラメータの与え方} % 担当2
      \subsubsection{バックプロパゲーションの改良} % 担当2

  \section{相互結合型ニューラルネットワークの情報処理}
    \subsection{相互結合型ニューラルネットワークの形態}
    \subsection{連想記憶}
    \subsection{ホップフィールドモデル}
      \subsubsection{2値ホップフィールドモデル}
      \subsubsection{連想記憶へのおう }
      \subsubsection{連続値ホップフィールドモデル}
      \subsubsection{最適化問題への応用}
      \subsubsection{連続値ホップフィールドモデルの改良}
    \subsection{ボルツマンマシン}
      \subsubsection{ボルツマンマシンの動作} % 担当3
      \subsubsection{ボルツマンマシンの学習}
      \subsubsection{ボルツマンマシンの特徴}

  \section{競合学習型ニューラルネットワークの方法処理}
    \subsection{認識機構の自己形成}
    \subsection{生体のトポロジカルマッピングのモデル}
    \subsection{コホーネンのモデル}
      \subsubsection{予備実験}
      \subsubsection{特徴抽出細胞の形成} % 担当4
      \subsubsection{コホーネンの学習則} % 担当4
      \subsubsection{コホーネンの自己組織化特徴マップのアルゴリズムとシミュレーション} % 担当4
      \subsubsection{応用例} % 担当4

  % やらない
  % \section{ニューラルネットワークにおける競合と協調}

  \section{ニューラルネットワーク研究の意義}
    \subsection{特徴を生かす}
      \subsubsection{研究の歴史}
      \subsubsection{生物内のニューラルネットワークと人工ニューラルネットワーク}
      \subsubsection{シナプスの可塑性と脳・神経系の可塑性}
      \subsubsection{教師あり学習と教師なし学習}
      \subsubsection{ニューロンコンピュータ}
      \subsubsection{融合化技術}
    \subsection{応用}
      \subsubsection{応用されてきた分野}
      \subsubsection{事例の完備性と適用有効範囲}
      \subsubsection{ブラックボックスモデルの利用環境への適合性}
    \subsection{脳科学への貢献}

\end{document}

