\documentclass[uplatex, 11pt, a4j, dvipdfmx]{jsarticle}

%------------------------------------------------------------------------
% プリアンブル読み込み
%------------------------------------------------------------------------
%------------------------------------------------------------------------
% usepackages
%------------------------------------------------------------------------
% 使用可能な色を増やす
\usepackage[dvipsnames,svgnames,x11names,dvipdfmx,table,xcdraw]{xcolor}
% 画像読込みを可能に
\usepackage[dvipdfmx]{graphicx}
% レイアウト変更用
\usepackage{geometry}
% 出力用紙サイズのみの変更
\usepackage{bxpapersize}
% セクションのサイズ等を変更
\usepackage{titlesec}
% テーブルのセル内で改行を可能に
\usepackage{tabularx}
% 画像の位置を書いた位置に矯正
\usepackage{here}
% url の表示を可能に
\usepackage{url}
% ソースコード表示用
\usepackage{listings, jlisting}
% otf フォントを使用可能に
\usepackage[uplatex]{otf}
% フォントエンコーディングをT1に変更し、欧文フォントのデフォルトを Latin Modern に変える
\usepackage[T1]{fontenc}
\usepackage{lmodern}

% 図を複数個同一環境にならべたときにキャプションを個別につけられるように
\makeatletter
\let\MYcaption\@makecaption
\makeatother
\usepackage{subcaption}
\captionsetup{compatibility=false}
\makeatletter
\let\@makecaption\MYcaption
\makeatother

% 数式系
\usepackage{amsmath, amssymb}
\usepackage{diffcoeff}
\usepackage{newtxmath, newtxtext}
\usepackage{bm}
\usepackage{ascmac}
% hyperlink
\usepackage[dvipdfmx, bookmarkstype=toc, colorlinks=true, linkcolor=cyan, pdfborder={0 0 0}, bookmarks=true, bookmarksnumbered=true, hypertexnames=false]{hyperref}
\usepackage{pxjahyper}
% 参照を便利に
\usepackage{cleveref}
% 参照した数式のみに数式番号をつける
\usepackage{autonum}
% 数学系いろいろ
\usepackage{mathtools}

% SI単位用
\usepackage{siunitx}                    % SI



%------------------------------------------------------------------------
% setting for listing
%------------------------------------------------------------------------
\lstset{%
    language={C},
    basicstyle={\small},%
    identifierstyle={\small},%
    commentstyle={\small\itshape},%
    keywordstyle={\small\bfseries},%
    ndkeywordstyle={\small},%
    stringstyle={\small\ttfamily},
    frame={tb},
    breaklines=true,
    columns=[l]{fullflexible},%
    numbers=left,%
    xrightmargin=0zw,%
    xleftmargin=3zw,%
    numberstyle={\scriptsize},%
    stepnumber=1,
    numbersep=1zw,%
    lineskip=-0.5ex%
}
\lstset{%
    language={tex},
    basicstyle={\small},%
    identifierstyle={\small},%
    commentstyle={\small\itshape},%
    keywordstyle={\small\bfseries},%
    ndkeywordstyle={\small},%
    stringstyle={\small\ttfamily},
    frame={tb},
    breaklines=true,
    columns=[l]{fullflexible},%
    numbers=left,%
    xrightmargin=0zw,%
    xleftmargin=3zw,%
    numberstyle={\scriptsize},%
    stepnumber=1,
    numbersep=1zw,%
    lineskip=-0.5ex%
}


%------------------------------------------------------------------------
% change the numbering \equation with (section.equation)
%------------------------------------------------------------------------
\makeatletter
    \renewcommand{\theequation} {%
        %\arabic{chapter}.\arabic{section}.\arabic{equation}}
        \arabic{section}.\arabic{equation}}
    \@addtoreset{equation}{section}
\makeatother



\crefname{figure}{図}{図}
\crefname{subfigure}{図}{図}
\crefname{table}{表}{表}
\crefname{subtable}{表}{表}
\crefname{equation}{式}{式}

\newtheorem{dfn}{定義}
\crefname{dfn}{定義}{定義}

\newtheorem{thm}{定理}
\crefname{thm}{定理}{定理}

\newtheorem{lem}{補題}
\crefname{lem}{補題}{補題}

\newtheorem{coro}{系}
\crefname{coro}{系}{系}

\newtheorem{prop}{命題}
\crefname{prop}{命題}{命題}

%\newtheorem{proof}{証明}
\crefname{proof}{証明}{証明}

\newtheorem{ex}{例}
\crefname{ex}{例}{例}

\newtheorem{rem}{注意}
\crefname{rem}{注意}{注意}

\crefformat{page}{#2#1#3{ページ}}%
\crefrangeformat{page}{#3#1#4{〜}#5#2#6{ページ}}%
\crefmultiformat{page}{#2#1#3{ページ}}%
        {, #2#1#3{ページ}}{, #2#1#3}{, #2#1#3{ページ}}%
\crefformat{section}{#2#1#3{節}}%
\crefrangeformat{section}{#3#1#4{〜}#5#2#6{節}}%
\crefmultiformat{section}{#2#1#3{節}}%
        {, #2#1#3{節}}{, #2#1#3}{, #2#1#3{節}}%
\crefformat{subsection}{#2#1#3{節}}%
\crefrangeformat{subsection}{#3#1#4{〜}#5#2#6{節}}%
\crefmultiformat{subsection}{#2#1#3{節}}%
        {, #2#1#3{節}}{, #2#1#3}{, #2#1#3{節}}%
\crefformat{subsubsection}{#2#1#3{節}}%
\crefrangeformat{subsubsection}{#3#1#4{〜}#5#2#6{節}}%
\crefmultiformat{subsubsection}{#2#1#3{節}}%
        {, #2#1#3{節}}{, #2#1#3}{, #2#1#3{節}}%

\crefformat{part}{{第}#2#1#3{部}}%
\crefformat{chapter}{{第}#2#1#3{章}}%

% fig
\newcommand{\fig}[5]{\begin{figure}[#4]\begin{center}\includegraphics[#5]{#1}\caption{#2}\label{fig:#3}\end{center}\end{figure}}

% tab
\newenvironment{tab}[4]{\begin{table}[#1]\centering \caption{#2}\label{tab:#4}\begin{tabular}{#3}}{\end{tabular}\end{table}}

% set
\def\Set#1{\Setdef#1\Setdef}
\def\Setdef#1|#2\Setdef{\left\{#1\,\;\mathstrut\vrule\,\;#2\right\}}%

% sets
\newcommand{\N}{\mathbb{N}}
\newcommand{\Z}{\mathbb{Z}}
\newcommand{\R}{\mathbb{R}}
\newcommand{\Q}{\mathbb{Q}}
\newcommand{\C}{\mathbb{C}}



%------------------------------------------------------------------------
% change font size of section and subsection title
%------------------------------------------------------------------------
%\titleformat*{\section}{\normalsize\bfseries}
%\titleformat*{\subsection}{\normalsize\bfseries}

%------------------------------------------------------------------------
% other setting
%------------------------------------------------------------------------
%\setlength{\columnsep}{3zw}
\setcounter{tocdepth}{3}                       % chenge the depth displayed in the table of contents




\title{代数学}
\author{}
\date{}

% -----------------------------------------------------------------------
% 本文
% -----------------------------------------------------------------------

\begin{document}

\maketitle

\section{整数の性質}
  \subsection{基礎}
    整数全体の集合$\Z$には加法、乗法という演算が定義れている。
    $\Z$上の加法と乗法は以下の性質を満たす。
    \begin{itemize}
      \item 任意の$a, b, c \in \Z$に対して
            \begin{equation} \begin{gathered}
              ( a + b ) + c = a + ( b + c ) \\
              ( a \cdot b ) \cdot c = a \cdot ( b \cdot c )
            \end{gathered} \end{equation}
            が成立する
      \item 任意の$a, b \in \Z$に対して
            \begin{equation} \begin{gathered}
              a + b = b + a \\
              a \cdot b = b \cdot a
            \end{gathered} \end{equation}
            が成立する
      \item 任意の$a, b, c \in \Z$に対して
            \begin{equation} \begin{gathered}
              ( a + b ) \cdot c = a \cdot c + b \cdot c \\
              a \cdot ( b + c ) = a \cdot b + a \cdot c
            \end{gathered} \end{equation}
            が成立する
      \item 加法と乗法にはそれぞれ単位元$0 \in \Z$と$1 \in \Z$が存在する
      \item 任意の$x \in \Z$に対して加法に関する逆元$-x \in \Z$が存在する
    \end{itemize}

    このような構造を代数的構造($\Z$は環という構造)という。

    さらに$\Z$は以下のような性質を満たす。
    \begin{itemize}
      \item 任意の$a, b \in \Z$に対して以下の3つのうちいずれか1つのみが成立する
        \begin{itemize}
          \item a > b
          \item a < b
          \item a = b
        \end{itemize}
      \item 任意の$a, b, c \in \Z$に対して$a < b$かつ$b < c$ならば$a < c$である
      \item 任意の$a, b, c \in \Z$に対して$a < b$ならば$a + c < b + c$である
    \end{itemize}

    このような構造を順序構造という。

    \begin{screen}
      \begin{prop}
        $a, b, c \in \Z$のとき、$a c = b c$かつ$c \neq 0$ならば$a = b$である
      \end{prop}
    \end{screen}
    \begin{proof}
      まず$a c = b c$の両辺に$b c$の加法逆元$-bc$を加える。
      すると
      \begin{equation} \begin{aligned}
                            a c - b c &= b c - b c \\
        \rightleftharpoons  a c - b c &= 0
      \end{aligned} \end{equation}
      となり、分配法則から
      \begin{equation} \begin{aligned}
        a c - b c &= (a - b) c
      \end{aligned} \end{equation}
      であるため
      \begin{equation} \begin{aligned}
                            a c - b c &= 0 \\
        \rightleftharpoons  (a - b) c &= 0  \label{eq:aster-1}
      \end{aligned} \end{equation}
      となる。
      \cref{eq:aster-1}より$a - b = 0$もしくは$c = 0$のどちらか一方が成立する。
      しかし、仮定より$c \neq 0$であるため
      \begin{equation}
        a - b = 0
      \end{equation}
      が成立する。
      両辺に$b$を加えると
      \begin{equation} \begin{aligned}
                            a - b + b &= 0 + b \\
        \rightleftharpoons  a + b - b &= b \\
        \rightleftharpoons  a + 0 &= b \\
        \rightleftharpoons  a &= b
      \end{aligned} \end{equation}
      となり、示された。
    \end{proof}

    \begin{screen}
      \begin{prop}
        $a \in \Z$のとき$0 \cdot a = 0$である。
      \end{prop}
      \begin{proof}
        零元の性質から$0 = 0 + 0$であるため
        \begin{equation} \begin{aligned}
          0 \cdot a = (0 + 0) \cdot a
        \end{aligned} \end{equation}
        であり、分配法則より
        \begin{equation} \begin{aligned}
          0 \cdot a &= (0 + 0) \cdot a \\
                    &= 0 \cdot a + 0 \cdot a
        \end{aligned} \end{equation}
        となる。
        ここで両辺に$0 \cdot a$の加法逆元$-(0 \cdot a)$を加えると、
        \begin{equation} \begin{aligned}
          0 \cdot a               &= 0 \cdot a + 0 \cdot a \\
          0 \cdot a - (0 \cdot a) &= 0 \cdot a + 0 \cdot a  - (0 \cdot a) \\
          0                       &= 0 \cdot a + 0
        \end{aligned} \end{equation}
        となり示された。
      \end{proof}
    \end{screen}

    このように様々な命題は代数的構造の定義から証明できる。
    しかし、手間がかかるため以降簡単なものは認めて議論を行う。

    \begin{screen}
      \begin{thm}
        \label{thm:division-theorem}
        除法の定理\\
        $a \in \Z, a \geq 0, b \in \N$とする。
        このとき次を満たす$q, r \in \Z$の組が唯一つ定まる。
        \begin{equation}
          a = b \cdot q + r \quad (a \leq r < b)
        \end{equation}
        このとき$q$のことを$a$を$b$で割った商、$r$のことを剰余と言う。

        この定理は整数をグループ分けしているとも言える。
      \end{thm}
    \end{screen}

    この定理は$(q, r) \in \Z^2$存在性と一意性の2つを主張している。
    除法の定理の証明には数学的帰納法を用いる。

    \begin{screen}
      自然数の性質(数学的帰納法の原理)
      \begin{description}
        \item[第1形式] $S \subset \N$のとき、$1 \in S$であり任意の$a \in S$に対して$a \in S \Rightarrow a + 1 \in S$が成立するならば$S = \N$である
        \item[第2形式] $S \subset \N$のとき、$1 \in S$でありある$a \in S$に対して$1, 2, \cdots, a \in S \Rightarrow a + 1 \in S$が成立するならば$S = \N$である
      \end{description}
    \end{screen}

    この自然数の性質を用いて数学的帰納法の正当性を示す。

    \begin{screen}
      \begin{thm}
        \label{thm:mathematical-induction}
        各自然数ごとの命題$P(n)$について
        \begin{enumerate}
          \item $P(1)$が真である \label{en:re-1}
          \item ある$a \in \N$に対して、$P(a)$が真ならば$P(a + 1)$も真である \label{en:re-2}
        \end{enumerate}
        の双方が真であるならば、任意の$m \in \N$に対して$P(m)$が真である。
      \end{thm}
      \begin{proof}
        $S := \Set{ n \in \N | P(n) }$とし、\cref{en:re-1}, \cref{en:re-2}ともに成立するとする。
        ここで\cref{en:re-1}から$1 \in S$であり、\cref{en:re-2}から任意の$a \in \N$に対して$a \in S$ならば$a + 1 \in S$である。
        したがって自然数の性質より
        \begin{equation}
          S = \N \text{であり}
        \end{equation}
        $S$の定義から任意の$n \in S$に対して$P(n)$が真となるため、任意の$n \in \N$に対しても$P(n)$が真となる。
      \end{proof}
    \end{screen}

    \cref{thm:division-theorem}を数学的帰納法を用いて証明する。
    \begin{proof}
      まず存在証明を行う。
      \begin{enumerate}
        \item $0 \leq a < b$のとき$ = 0, r = a$とすれば
          \begin{equation}
           a = b \cdot q + r \rightleftharpoons a = r
          \end{equation}
          となり、存在することが示された。
        \item $a \geq b$とし、$0 \leq k \leq a - 1$に対して$q, r$が存在すると仮定する。
              このとき$a \geq b, b > 0$であることから
              \begin{equation}
                0 \leq a - b \leq a - 1
              \end{equation}
              が成立する。
              よって仮定より
              \begin{equation}
                a - b = b \cdot q' + r'
              \end{equation}
              を満たす$q', r'$が存在する。
              この式を変形すると
              \begin{equation} \begin{aligned}
                a - b = b \cdot q' + r' \\
                \rightleftharpoons a = b \cdot q' + b + r' \\
                \rightleftharpoons a = b \cdot (q' + 1) + r'
              \end{aligned} \end{equation}
              となり、$q = q' + 1, r = r'$としたときに
              \begin{equation}
                a  = b \cdot q + r
              \end{equation}
              が成立する。
              このとき$q, r$ともに自然数であるため、存在することが示された。
      \end{enumerate}
      以上より、\cref{thm:mathematical-induction}から任意の$a \in \Set{ x \in \Z | 0 \leq x }, b \in \N$に対して
      \begin{equation}
        a = b \cdot q + r \quad (a \leq r < b)
      \end{equation}
      を満たす$q, r$が存在する。

      次に$q, r$の一意性を示す。
      ある$a, q_1, q_2 \in \Set{ x \in \Z | 0 \leq x }, b \in \N, r_1, r_2 \in \Set{ x \in \Z | 0 \leq x < b }$に対して
      \begin{equation}\begin{gathered}
        a = b \cdot q_1 + r_1, \quad a = b \cdot q_2 + r_2
      \end{gathered}\end{equation}
      が成立するとする。
      ここで$q_1 \geq q_2 + 1$とすると
      \begin{equation} \begin{aligned}
        b \cdot q_1 + r_1 \geq b \cdot q_1 \geq b \cdot (q_2 + 1) \\
        \rightleftharpoons b \cdot q_1 + r_1 \geq b \cdot q_1 \geq b \cdot q_2 + b \\
        \rightleftharpoons b \cdot q_1 + r_1 \geq b \cdot q_2 + b
      \end{aligned} \end{equation}
      となり
      \begin{equation}
        b \cdot q_2 + b > b \cdot q_2 + r_2
      \end{equation}
      であるため
      \begin{equation} \begin{aligned}
        b \cdot q_1 + r_1 \geq b \cdot q_2 + b > b \cdot q_2 + r_2 \\
        \rightleftharpoons b \cdot q_1 + r_1 > b \cdot q_2 + r_2 \\
      \end{aligned} \end{equation}
      が成立する。
      仮定より
      \begin{equation} \begin{aligned}
        b \cdot q_1 + r_1 > b \cdot q_2 + r_2 \rightleftharpoons a > a
      \end{aligned} \end{equation}
      であり矛盾するため$q_1 < q_2 + 1$である。
      $q_1 + 1 > q_2$とすると
      \begin{equation} \begin{aligned}
        b \cdot q_1 + r_1 < b \cdot q_1 + b \\
        \rightleftharpoons b \cdot q_1 + r_1 < b \cdot (q_1 + 1)
      \end{aligned} \end{equation}
      となり
      \begin{equation}
        b \cdot (q_1 + 1) \leq b \cdot q_2 \leq b \cdot q_2 + r_2
      \end{equation}
      であるため
      \begin{equation} \begin{aligned}
        b \cdot q_1 + r_1 < b \cdot (q_1 + 1) \leq b \cdot q_2 \leq b \cdot q_2 + r_2 \\
        \rightleftharpoons b \cdot q_1 + r_1 <  b \cdot q_2 + r_2
      \end{aligned} \end{equation}
      が成立する。
      仮定より
      \begin{equation} \begin{aligned}
        b \cdot q_1 + r_1 < b \cdot q_2 + r_2 \rightleftharpoons a < a
      \end{aligned} \end{equation}
      であり矛盾するため$q_1 + 1 > q_2$である。

      以上より$q_1, q_2$に対して
      \begin{equation}\begin{gathered}
        q_1 < q_2 + 1, \quad q_1 + 1 > q_2
      \end{gathered}\end{equation}
      が成立しなければならないため$q_1 = q_2$となる。
      さらに$r_1 = a - b \cdot q_1, r_2 = a - b \cdot q_2$なので$r_1 = r_2$となり一意性が示された。

      よって\cref{thm:division-theorem}が正しいことが示された。
    \end{proof}

    \begin{screen}
      \begin{thm}
        \label{thm:division-theorem-2}
        除法の定理の拡張\\
        $a \in \Z, a \geq 0, b \in \Z - \{0\}$とする。
        このとき
        \begin{equation}
          a = b \cdot q + r \quad (a \leq r < |b|)
        \end{equation}
        を満たす$q, r \in \Z$の組が唯一つ定まる。
      \end{thm}
    \end{screen}

    \cref{thm:division-theorem-2}を示す。
    \begin{proof}
      $b$が$0$より大きい場合については\cref{thm:division-theorem}によって示されているため、$b < 0$の場合について示せばよい。

      $b < 0$の場合も\cref{thm:division-theorem}と同様に数学的帰納法によって示すことができる。%TODO
    \end{proof}

  \subsection{約数}
    \begin{screen} \begin{dfn}
      $a, b \in \Z$とする。
      このとき
      \begin{equation}
        \text{$a$は$b$で割り切れる} :\Leftrightarrow \exists q \in Z s.t. a = bq
      \end{equation}
      と定義する。
      そして$a$が$b$で割り切れることを$b \mid a$と書く。
    \end{dfn} \end{screen}
    \begin{screen} \begin{prop}
      $a, b, c \in \Z$としたとき以下の3つが成立する。
      \begin{enumerate}
        \item $b \mid a \land b \mid c$ならば$b \mid (a + c)$
        \item $b \mid a$ならば$\forall x \in \Z s.t. b \mid ax$
        \item $c \mid b \land b \mid a$ならば$c \mid a$
      \end{enumerate}
    \end{prop} \end{screen}



    % ここまでで ノート p.14 (such thatの書きかた)



\end{document}

